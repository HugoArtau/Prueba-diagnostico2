\documentclass[11pt, addpoints, answers]{exam}
\usepackage[spanish]{babel}
\usepackage{amsmath}
\usepackage{amssymb}
\usepackage{graphicx}
\usepackage{tabularx}
\usepackage{ragged2e}
\usepackage{geometry}
\usepackage{booktabs}\pointpoints{punto}{puntos}

% --- AJUSTE DE MARGENES ---
\geometry{
	hmargin={1in, 1in},
	vmargin={1.2in, 1in},
}
% --- CONFIGURACIÓN DE PÁGINA Y ENCABEZADO FINAL ---
\pagestyle{headandfoot}
% 1. Definición Ultra-Robusta del Encabezado (soluciona Overfull \hbox)
\renewcommand{\firstpageheadrule}{%
	\dimen0=\tabcolsep
	\makebox[\textwidth]{%
		\hspace{-\dimen0}\makebox[\textwidth+2\dimen0]{%
			\textbf{Nombre del estudiante:} \enspace\hrulefill \hspace{2em}
			\textbf{Grupo:} \enspace\hrulefill
		}%
	}\vspace{1ex}\hrule
}
% 2.  Se define el contenido del encabezado de la universidad (centrado)
\firstpageheader{}{\centering\textbf{Examen Diagnóstico de Matemática. Batería 2}\\\textbf{Universidad Estatal Guayaquil}}{}

% 3. Define el pie de página
\firstpagefooter{}{Página \thepage\ de \numpages}{}
\runningheader{Diagnóstico Matemática}{}{Pág. \thepage}
\runningfooter{}{}{}

\begin{document}
	
	% --- NUEVA SECCIÓN PARA DATOS DEL ALUMNO ---
	% Usamos \makebox para alinear el Nombre a la izquierda y el Grupo a la derecha.
	\makebox[\textwidth]{
		\textbf{Nombre:} \enspace\hrulefill \hspace{4em}  
		\textbf{Grupo:} \enspace\hrulefill 
	}
	
	
	
	\vspace{1 cm}
	
		% COMIENZA EL EXAMEN (Todas las preguntas)
	\begin{questions}
	\question[1] \textbf{	Considerando las siguientes funciones, calcula sus respectivas imágenes para $x = -1$; $x = 0 $; $x = 1 $; $x =\sqrt{2}$ }
		\begin{parts}
		\part $f(x)=4x^2 + 2 $
		\part $g(x)= -2x^{2}+2$
		\part $h(x)=\sqrt{2} - 2x^{2}$
		\end{parts}
		\question[1] \textbf{Determina si las siguientes funciones son inyectivas.}
		\begin{parts}
		\part $f(x)=5x+\frac{1}{2}$
		\part $g(x)=-4x^{2}+6$
		\part $h(x)=\sqrt{x}-3$
		\part $k(x)=x^{2}+1$
		\end{parts}
			\question[1] \textbf{Dadas las siguientes funciones $g(x)=4x^{2}+\dfrac{1}{2}$  y   $f(x)=5x^{2}-\dfrac{1}{2}$. Calcule.}
			\begin{parts}
				\part $f(x)+g(x)$
				\part $f(x)-g(x)$
				\part $(g\circ f)$
				\part $\frac{g(x)}{f(x)}$
			\end{parts}
			\question[1]\textbf{De la función $f(x)=x^{2}-2x-8$ diga cuáles son sus raíces:}
			\begin{parts}
			\part$x_{1}=2$ $x_{2}=4$
			\part$x_{1}=-2$ $x_{2}=-4$
			\part$x_{1}=-2$ $x_{2}=4$
			\part$x_{1}=-2$ $x_{2}=2$
			\end{parts}
			\question[1]\textbf{Asocia cada función con su gráfica.}
			\begin{parts}
			\part $f(x)=\log_2(x)$
			\part $g(x)=5$
			\part $h(x)= \left| \frac{1  }{2}x+5 \right|$
			\part $k(x)=-\dfrac{1}{2}x+5$
			\part $m(x)=x^{2}-2x-8$
			\part $n(x)=2^{x}$
			\part $p(x)=2x$
			\part $q(x)=-x^{2}-2x+8$
		\end{parts}
	\end{questions}	
	\begin{figure*}[h]
		\includegraphics[width=0.5\linewidth]{Figuras/Fig_1.jpg}
	
		\label{fig:fig1}
	\end{figure*}
	
		
		
		
\end{document}